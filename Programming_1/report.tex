\documentclass{ctexart}

\usepackage{graphicx}
\usepackage{amsmath}
\usepackage{float}

\title{Programming 1}

\author{洪晨瀚 \\ 信息与计算科学 3200300133}

\begin{document}

\maketitle

\section*{Question B}
\begin{flushleft}
  用二分法找出$f(x)$在定义域中的根:\\
  a. \verb|input| : $f(x) = x^{-1} - \tan x$ , $[0,\frac{\pi}{2}]$ \\
  \quad \verb|output| : $0.860334$ \\
 
  b. \verb|input| : $f(x) = x^{-1} - 2^x$ , $[0,1]$ \\
  \quad \verb|output| : $0.641186$ \\

  c. \verb|input| : $f(x) = 2^{-x} + e^x + 2\cos x - 6$ , $[1,3]$ \\
  \quad \verb|output| : $1.82938$ \\

  d. \verb|input| : $f(x) = \frac{x^3 + 4x^2 + 3x + 5}{2x^3 - 9x^2 + 18x -2}$ , $[0,4]$ \\
  \quad \verb|output| : $0.117877$ \\ 
\end{flushleft}

\section*{Question C}
\begin{flushleft}
  用牛顿法找出$f(x)$在两个初值附近的根:\\
  a. \verb|input| : $f(x) = \tan x - x$ , $x_0 = 4.5$ \\
  \quad \verb|output| : $4.49341$ \\
 
  b. \verb|input| : $f(x) = \tan x - x$ , $x_1 = 7.7$ \\
  \quad \verb|output| : $7.72525$ \\
\end{flushleft}

\clearpage
\section*{Question D}
\begin{flushleft}
  用割线法找出$f(x)$的根:\\
  a. \verb|input| : $f(x) = \sin {\frac{x}{2}} -1$ , $x_0 = 0$ , $x_1 = \frac{\pi}{2}$\\
  \quad \verb|output| : $3.14159$ \\
 
  b.  \verb|input| : $f(x) = e^x - \tan x$ , $x_0 = 1$ , $x_1 = 1.4$\\
  \quad \verb|output| : $1.30633$ \\
  
  c.  \verb|input| : $f(x) = x^3 - 12x^2 + 3x + 1$ , $x_0 = 0$ , $x_1 = -0.5$\\
  \quad \verb|output| : $-0.188685$ \\
\end{flushleft}

\section*{Question E}
\begin{flushleft}
  $f(h) = L (0.5\pi r^2 - r^2\arcsin \frac{h}{r} - h\sqrt{r^2 - h^2}) - V$ \\
  $L=10 , r=1 , V=12.4$ \\
  二分法 : 定义域为$[0,1]$ , 根为 $0.166166$。 \\
  牛顿法 : 初值为$0$ , 根为 $0.166166$。\\
  割线法 : 初值分别为$0$和$1$ , 根为 $0.166166$。\\
\end{flushleft}

\clearpage
\section*{Question F}
\begin{flushleft}
  $f(\alpha) = A\sin\alpha \cos\alpha + B\sin^2\alpha - C\cos\alpha - E\sin\alpha$ \\
  $A = l \sin\beta$ \\
  $B = l \cos\beta$ \\
  $C = (h + 0.5D)\sin\beta - 0.5D\tan\beta$ \\ 
  $E = (h + 0.5D)\cos\beta - 0.5D$ \\
\end{flushleft}

\subsection*{a. 用牛顿法证明$\alpha = 33^{\circ}$ }
\begin{flushleft}
  \quad $l=89 , h=49 , D=55 , \beta=11.5^{\circ}$ \\
  \quad \verb|output| : $-11.5^{\circ}$ \\
\end{flushleft}

\subsection*{b. 用牛顿发找$\alpha$,且初值为$33^{\circ}$}
\begin{flushleft}
  \quad $l=89 , h=49 , D=30 , \beta=11.5^{\circ}$ \\
  \quad \verb|output| : $-11.5^{\circ}$ \\
\end{flushleft}

\subsection*{c. 用割线法找$\alpha$,其中一个初值为$33^{\circ}$,另一个初值为$333^{\circ}$}
\begin{flushleft}
  \quad $l=89 , h=49 , D=55 , \beta=11.5^{\circ}$ \\
  \quad \verb|output| : $168.5^{\circ}$ \\
\end{flushleft}
  
\end{document}
